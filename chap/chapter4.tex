\chapter{一些例子}
\section{各种例子}
\subsection{插图表格}
\begin{figure}[htbp]
    \centering\includegraphics[width=5cm,height=1.32cm]{figures/logo3.pdf}
    \caption[中英校名]{中英校名}
\end{figure}
\begin{table}[htbp]
    \noindent\begin{minipage}{0.5\textwidth}
        \centering
        \caption{并排子表格}
        \label{tab:parallel1}
        \begin{tabular}{p{2cm}p{2cm}}
            \toprule[1.5pt]
            姓名 & 性别 \\\midrule[1pt]
            李狗蛋 & 女 \\\bottomrule[1.5pt]
        \end{tabular}
    \end{minipage}
    \begin{minipage}{0.5\textwidth}
        \centering
        \caption{并排子表格}
        \label{tab:parallel2}
        \begin{tabular}{p{2cm}p{2cm}}
            \toprule[1.5pt]
            姓名 & 性别 \\\midrule[1pt]
            张狗蛋 & 女 \\\bottomrule[1.5pt]
        \end{tabular}
    \end{minipage}
\end{table}
\begin{table}[htbp]
    \centering
    \caption{并排子表格}
    \label{tab:subtable}
    \subtable[第一个子表格]
    {
        \begin{tabular}{p{2cm}p{2cm}}
            \toprule[1.5pt]
            姓名 & 性别 \\\midrule[1pt]
            田狗蛋 & 男 \\\bottomrule[1.5pt]
        \end{tabular}
    }
    \hskip2cm
    \subtable[第二个子表格]
    {
        \begin{tabular}{p{2cm}p{2cm}}
            \toprule[1.5pt]
            姓名 & 性别 \\\midrule[1pt]
            李狗蛋 & 女 \\\bottomrule[1.5pt]
        \end{tabular}
    }
\end{table}

\subsection{数学环境}
下面是几个数学公式的例子:\par
\begin{equation}
    \begin{aligned}
        P\{S_n \leq t\} &= \int_{-\infty}^{+\infty}f_{S_n}dt \notag \\
                       &= \int_0^t\frac{\lambda(\lambda u)^{n-1}}{(n-1)!}e^{-\lambda u}du \\
                       &\xlongequal{令 \lambda u=x} \frac{1}{(n-1)!}\int_0^{\lambda t}x^{n-1}e^{-x}dx\\
                       &=\frac{-1}{(n-1)!}(e^{-x}x^{n-1}{\Big|}_0^{\lambda t}-\int_0^{\lambda t}e^{-x}dx^{n-1})\\
                       &=\frac{-1}{(n-1)!}e^{-x}x^{n-1}{\Big|}_0^{\lambda t}+\frac{1}{(n-2)!}\int_0^{\lambda t}e^{-x}x^{n-2}dx
    \end{aligned}
\end{equation}\par
再来几个:
\begin{equation}
    \begin{aligned}
        \lambda &=\left (1+\frac{\left(\frac{\bar{X}-\bar{Y}}{\sqrt{((\frac{1}{n}+\frac{1}{m})\sigma^2)}}\right)^2}{\left(\sqrt{\frac{\sum\limits_{i=1}^n(X_i-\bar{X})^2+\sum\limits_{i=1}^m(Y_i-\bar{Y})^2}{(m+n)\sigma^2}}\right)^2(m+n-2)}\right)^{\frac{n+m}{2}} \\ \notag
            &=\left(1+\frac{T^2}{n+m-2}\right)^{\frac{n+m}{2}}\\
        其中\quad T^2 &=\left(\frac{\frac{\bar{X}-\bar{Y}}{\sqrt{((\frac{1}{n}+\frac{1}{m})\sigma^2)}}}{{\sqrt{\frac{\sum\limits_{i=1}^n(X_i-\bar{X})^2+\sum\limits_{i=1}^m(Y_i-\bar{Y})^2}{(m+n)\sigma^2}}}}\right)^2
    \end{aligned}
\end{equation}
