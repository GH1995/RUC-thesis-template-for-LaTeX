\chapter{\RUCThesis\ 介绍}
\section{必要的宏包}
本模板中包含的宏包如下表所示:
\begin{table}[htb]
    \centering
    \begin{minipage}[t]{0.8\linewidth}
        \caption[必要宏包]{本模板中包含的宏包,当然这些必须安装。其实这些在你的\LaTeX\ 里面已经有了。其实这还是一个简单三线表的例子,其实这还是一个长表格标题的例子,当然还是一个表格中插脚注的例子}
        \label{tab:template-files}
        \begin{tabular*}{\linewidth}{cccccc}
            \toprule[1.5pt]
            \multicolumn{6}{c}{\sf 宏包文件}\\ \midrule[1pt]
            ctexbook & geometry & hyperref & graphicx\footnote{插图宏包} & titletoc & ifxetex \\
            ifthen  & calc & lscape\footnote{页面横向放置宏包}   & multicol & color   & pstricks\\
            \bottomrule[1.5pt]
        \end{tabular*}
    \end{minipage}
\end{table}
\section{必要的字体}
字体命令为{\tt\textbackslash rm}衬线字{\tt\textbackslash sf}非衬线字{\tt\textbackslash tt}等线体。\par
本模板中包含的字体如下表所示:
\begin{longtable}[c]{c*{2}{l}}
    \caption{必要的字体}\label{tab:performance}\\
    \toprule[1.5pt]
    字体 & PostScript名称\\\midrule[1pt]
    \endfirsthead
    \caption[]{必要的字体(续)}\\
    \toprule[1.5pt]
    字体 & PostScript名称\\\midrule[1pt]
    \endhead
    \hline
    \multicolumn{2}{r}{续下页}
    \endfoot
    \endlastfoot
    Times New Roman & TimesNewRomanPSMT \\
    Arial &  ArialMT \\
    Courier New & CourierNewPSMT\\
    宋体 & SimSun \\
    黑体 & SimHei \\
    仿宋 & FangSong \\
    方正小标宋\footnotemark & FZXBSJW–GB1-0 \\
    \bottomrule[1.5pt]
\end{longtable}
\footnotetext{只有在制作封皮的{\tt cover.tex}中用到。}


\section{编译源文件}
如果已经有ructhesis.cls文件的可以直接使用。
因为2015年12月正式实施了新的参考文献著录国标,新样式还在开发,先借用了一个(可能会报错)。 \par
这里我们使用xelatex作为引擎,在main.tex文件下使用如下命令:\\
{\tt
\$ xelatex main.tex\\
\$ bibtex main.tex\\
\$ xelatex main.tex\\
\$ xelatex main.tex\\}
想编译模板文件和生成手册的可以执行下述代码:\\
生成模板文件{\tt ructhesis.cls}\\
{\tt\$ latex ructhesis.ins}\\
生成手册{\tt ructhesis.pdf\\
\$ xelatex ructhesis.dtx\\
\$ makeindex -s gind.ist -o ructhesis.ind ructhesis.idx\\
\$ makeindex -s gglo.ist -o ructhesis.gls ructhesis.glo\\
\$ xelatex ructhesis.dtx\\
\$ xelatex ructhesis.dtx}\par
下图的文件是完整的{\tt RUCThesis}文件,同时下图也是横放大表格或者图片的例子。

\begin{landscape}
    \begin{figure}[htbp]
        \centering
        \tikzstyle{every node}=[anchor=west]
        \begin{tikzpicture}[%
                grow via three points={one child at (0.5,-0.7) and
            two children at (0.5,-0.7) and (0.5,-1.4)},
            edge from parent path={(\tikzparentnode.south) |- (\tikzchildnode.west)}]
            \node {\RUCThesis\ }
            child { node {chap}
            child { node {chapter1.tex\quad \% 章节文件}}
            child { node {...}}
            child { node {appendix\_{}1.tex\quad \% 附录}}}
            child [missing] {}
            child [missing] {}
            child [missing] {}
            child { node {cover.tex\quad \% 封面文件}}
            child { node {figures}
            child { node {logo.pdf\quad \% 图像}}
            child { node {...}}}
            child [missing] {}
            child [missing] {}
            child { node {format}
            child { node {acknowledge.tex\quad \% 致谢}}
            child { node {authorization.tex\quad \% 授权书影印件}}
            child { node {cabstractpage.tex\quad \% 中文摘要}}
            child { node {eabstractpage.tex\quad \% 英文摘要}}
            child { node {Originality.tex\quad \% 独创性声明}}}
            child [missing] {}
            child [missing] {}
            child [missing] {}
            child [missing] {}
            child [missing] {}
            child { node {main.tex\quad \% 主文件}}
            child { node {ref}
            child { node {ruc.bst\quad \% 参考文献样式}}
            child { node {yourbib.bib\quad \% 参考文献数据库}}}
            child [missing] {}
            child [missing] {}
            child { node {ructhesis.cls\quad \% \RUCThesis\ 文档类}};
        \end{tikzpicture}
        \caption{\RUCThesis\ 文件目录}
    \end{figure}
\end{landscape}


